\section{Introduction}\label{introduction}

\begin{frame}{Who am I?}

\begin{itemize}
\itemsep1pt\parskip0pt\parsep0pt
\item
  Jeffrey Skonhovd
\item
  Works at FTN Financial
\item
  Twitter: @jskonhovd
\item
  Github: jskonhovd
\end{itemize}

\end{frame}

\begin{frame}{Overview}

\begin{itemize}
\itemsep1pt\parskip0pt\parsep0pt
\item
  What is Machine Learning?

  \begin{itemize}
  \itemsep1pt\parskip0pt\parsep0pt
  \item
    Machine Learning is the study of computer algorithms that improve
    automatically through experience.
  \end{itemize}
\item
  How should I go about learning Machine Learning?

  \begin{itemize}
  \itemsep1pt\parskip0pt\parsep0pt
  \item
    MOOCs
  \item
    Don't get caught up in the implementations.
  \end{itemize}
\item
  Tools

  \begin{itemize}
  \itemsep1pt\parskip0pt\parsep0pt
  \item
    WEKA
  \item
    scikit-learn
  \end{itemize}
\end{itemize}

\end{frame}

\section{Machine Learning}\label{machine-learning}

\begin{frame}{Types}

\begin{itemize}
\itemsep1pt\parskip0pt\parsep0pt
\item
  Supervised Learning
\item
  Supervised Learning is \ldots{}
\item
  Unsupervised Learning
\item
  Unsupervised Learning is \ldots{}
\item
  Reenforcement Learning
\item
  Reenforcement Learning is \ldots{}
\end{itemize}

\end{frame}

\begin{frame}{Some Boring, but important Definitions.}

\begin{itemize}
\itemsep1pt\parskip0pt\parsep0pt
\item
  Inductive Bias
\item
  The inductive bias of a learning algorithm is the set of assumptions
  that the learner uses to predict outputs given inputs that it has not
  encountered.
\item
  Occam's Razor assumes that the hypotheses with the fewest assumptions
  should be selected.
\item
  Cross-validation
\item
  The basic idea of Cross-validation to leave out some of the data when
  fitting the model.
\end{itemize}

\end{frame}

\begin{frame}{Scikit-learn}

\begin{itemize}
\itemsep1pt\parskip0pt\parsep0pt
\item
  Scikit-learn is a set of simple and efficient tools for data mining
  and data analysis.
\item
  Uses Python!!!
\item
  \url{http://scikit-learn.org/}
\end{itemize}

\end{frame}

\section{Supervised Learning:
Scikit-learn}\label{supervised-learning-scikit-learn}

\begin{frame}{Decision Trees}

\begin{itemize}
\itemsep1pt\parskip0pt\parsep0pt
\item
  Decision Tree learning is a method for approximating discrete-valued
  target functions, in which the learned function is represented a
  decision tree.
\item
  Maximize Information Gain
\item
  Information Gain measures how well a given attribute separates the
  training examples according to their target classifcation.
\end{itemize}

\end{frame}

\begin{frame}[fragile]{Decision Trees: Example}

\begin{verbatim}
import numpy as np
import pylab as pl
from sklearn.datasets import load_iris
from sklearn.tree import DecisionTreeClassifier
# Parameters
# Load data
iris = load_iris()
clf = DecisionTreeClassifier()
X = iris.data[:, [1, 2]]
y = iris.target
clf = clf.fit(X, y)
plotCustom(X, y, [1, 2], clf)`
\end{verbatim}

\end{frame}

\begin{frame}[fragile]{kNN: Example}

\begin{verbatim}
from sklearn import neighbors
import numpy as np
import pylab as pl
from sklearn import cross_validation
from sklearn.datasets import load_iris

iris = load_iris()

X = iris.data[:, [1, 2]]
y = iris.target

clf = neighbors.KNeighborsClassifier(3, 'distance')

plotCustom(X, y, [1,2], clf)
\end{verbatim}

\end{frame}

\begin{frame}[fragile]{SVM}

\begin{verbatim}
from sklearn import svm
import numpy as np
import pylab as pl
from sklearn import cross_validation
from sklearn.datasets import load_iris

iris = load_iris()

X = iris.data[:, [1, 2]]
y = iris.target
C = 1.0
clf = svm.SVC(kernel='linear', C=C)
rbf_svc = svm.SVC(kernel='rbf', gamma=0.7, C=C)
poly_svc = svm.SVC(kernel='poly', degree=3, C=C)
lin_svc = svm.LinearSVC(C=C)
clf.fit(X,y)
rbf_svc.fit(X,y)
poly_svc.fit(X,y)
lin_svc.fit(X,y)
plotCustom(X, y, [1,2], rbf_svc)
plotCustom(X, y, [1,2], poly_svc)
plotCustom(X, y, [1,2], lin_svc)
\end{verbatim}

\end{frame}

\section{Unsupervised Learning:
Scikit-learn}\label{unsupervised-learning-scikit-learn}

\begin{frame}[fragile]{kMeans}

\begin{verbatim}
from time import time
import numpy as np
import pylab as pl
from sklearn.cluster import KMeans
from sklearn.datasets import load_digits
from sklearn.decomposition import PCA
from sklearn.preprocessing import scale
from sklearn.datasets import load_iris

iris = load_iris()

X = iris.data[:, [2, 3]]
y = iris.target
n_digits = len(np.unique(y))
kmeans = KMeans(init='k-means++', n_clusters=n_digits, n_init=10)
kmeans.fit(X)
kmeans_plots(X,y,[2, 3],kmeans)
\end{verbatim}

\end{frame}

\section{Conclusion}\label{conclusion}

\begin{itemize}
\itemsep1pt\parskip0pt\parsep0pt
\item
  Resources
\end{itemize}
